\documentclass[10pt]{report}
\usepackage{epsf}
\usepackage{amsmath}
\usepackage{amssymb}
\usepackage{palatino}
\usepackage[dvips]{graphics}
\usepackage{fancyhdr}
\usepackage{epsfig}
\usepackage{multirow}
\usepackage{multicol}
\usepackage{cancel}
\usepackage{hyperref}
\usepackage{longtable}
\parindent 0in
\parskip 1ex
\oddsidemargin  0in
\evensidemargin 0in
\textheight 8.5in
\textwidth 6.5in
\topmargin -0.25in

\pagestyle{fancy}
\lhead{\bf BME590.06: Medical Software Design}
\rhead{\bf Palmeri \& Kumar (Fall 2017)}
\cfoot{\thepage}


\begin{document}
\section*{Assignment \#01: Setup Course Tools \& Git Fundamentals}

{\bf DUE:} Wednesday, 2016-09-17 before class.

\begin{enumerate}

\item Create an account on GitHub (\url{https://github.com}).

\item Download and install \verb+git+ at \url{https://git-scm.com}.  We will be
using Git Bash, \emph{not} a GUI client.

\item Setup an SSH key to seamlessly push/pull to/from your GitHub repositories:
\url{https://help.github.com/articles/connecting-to-github-with-ssh/}

\item Download and install \verb+python3+ at \url{https://www.python.org/}. Be sure to install Python 3.6, \emph{not} Python 2.7.
Note that if you are using Windows, you should consider either:
\begin{itemize}
  \item Installing and using the Ubuntu Linux Subsystem (Windows 10), and
  running \verb+python3+ from within that environment, or
  \item Install Conda python from \url{https://www.continuum.io/downloads}.
\end{itemize}
Using ``vanilla'' python on Windows can have challenges with importing some
packages, such as \verb+numpy+, which do not exist in compiled wheels for
Windows.

\item You will want a code writing environment that makes life easier for you as
your projects get more complex.  Options include:
\begin{itemize}
  \item VIM [\url{http:\\www.vim.org}] (ideal for terminal usage)
  \item GitHub Atom [\url{https://atom.io/}]
  \item Visual Studio Code [\url{https://code.visualstudio.com/}]
  \item PyCharm [\url{https://www.jetbrains.com/pycharm/}]  (full-featured IDE)
\end{itemize}

\item Never used git before?  Start with these resources:
\begin{itemize}
    \item \url{https://try.github.io/}
    \item \url{https://www.codecademy.com/learn/learn-git}
    \item \url{https://www.git-tower.com/learn/cheat-sheets/vcs-workflow}
    \item \url{http://gitimmersion.com/}
    \item \url{https://www.atlassian.com/git/tutorials/comparing-workflows#feature-branch-workflow}
\end{itemize}

\item Familiar with git (or just completed the exercises above)?  Give this a
try: \url{http://learngitbranching.js.org/}

\item Having trouble?  We'll be reviewing some of these tools in lecture on
Thursday.  Also checkout the Duke Co-Lab, which hosts regular office hours and
has an online Slack team: \url{https://colab.duke.edu/}  We have a specific channel on there for this class, including:
\begin{itemize}
  \item \verb+#git+
  \item \verb+#python+
\end{itemize}

\item Learning Objectives:
\begin{itemize}
	\item Create a git repository on your local computer.
	\item Create a local file, then add and commit it to your local repository.
	\item Edit your local file, adding and committing those edits.
	\item Create a remote repository on GitHub that has the same name as your local repository.
	\item Add the remote repository (origin) URL to your local repository.
	\item Push your local repostiory to GitHub.
	\item Create a local branch, create/add/commit a new file.
	\item Merge new local branch commit(s) into local master.
	\item Push updated master branch to GitHub.
\end{itemize}

\end{enumerate}

\end{document}
