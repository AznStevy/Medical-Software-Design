\documentclass[10pt]{report}
\usepackage{epsf}
\usepackage{amsmath}
\usepackage{amssymb}
\usepackage{palatino}
\usepackage[dvips]{graphics}
\usepackage{fancyhdr}
\usepackage{epsfig}
\usepackage{multirow}
\usepackage{multicol}
\usepackage{cancel}
\usepackage{hyperref}
\usepackage{longtable}
\parindent 0in
\parskip 1ex
\oddsidemargin  0in
\evensidemargin 0in
\textheight 8.5in
\textwidth 6.5in
\topmargin -0.25in

\pagestyle{fancy}
\lhead{\bf BME590.06: Medical Software Design}
\rhead{\bf Palmeri \& Kumar (Fall 2017)}
\cfoot{\thepage}

\usepackage{hyperref}

\begin{document}

\section*{Final Projects}
{\bf Final project DUE:} --/--/--- (when our final exam block time would end) \\
{\bf RFC Draft DUE:} 11/20/2017 5:00PM EST 

\subsection*{Overview}
The final project in this class will require your team to leverage the industry-standard skills you've learned over this semester to design and implement a software system to acquire a signal, classify it using a cloud-based web service that you build (using deep learning or other machine learning approaches), and display the performance and outcomes of your system. This final project is purposely open-ended to allow groups to tailor this to their areas of interest. It is expected that your team will follow proper professional software development and design conventions taught in this class (git feature-branch workflow, continious integration, unit testing, PEP8, etc. as detailed in sections below). At minimum, your system must be comprised of:

\begin{itemize}
	\item A client device (the Raspberry Pi or equivalent) that will acquire a signal (image, biometric, etc), perform necessary preprocessing steps, and then issue a RESTful API request to your cloud service for further processing.
	\item A cloud-based web service that exposes a well-crafted RESTful API that will implement a deep-learning model to classify the signal into a labeled bucket (for example, high/low cancer risk).
\end{itemize}

\subsection*{RFC Engineering Design Document}
As discussed in class, before you start implementing your project, you and your team will be expected to flesh out a request for comments (RFC) document that will document the design, architecture, APIs, and edge cases for your system in advance. This document is traditionally reviewed (and commented on) by your colleagues in industry, and will be throughly reviewed by course instructors and TAs in this class. You and your team should also throughly review the RFC to ensure everyone is on the same page about what is going to be built (and in what way using what technologies, etc). 

You can find the Engineering RFC template at \underline{\href{http://rfc.suyash.io}{rfc.suyash.io}}.

\end{document}

