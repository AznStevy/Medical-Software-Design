\documentclass[10pt]{report}
\usepackage{epsf}
\usepackage{amsmath}
\usepackage{amssymb}
\usepackage{palatino}
\usepackage[dvips]{graphics}
\usepackage{fancyhdr}
\usepackage{epsfig}
\usepackage{multirow}
\usepackage{multicol}
\usepackage{cancel}
\usepackage{hyperref}
\usepackage{longtable}
\parindent 0in
\parskip 1ex
\oddsidemargin  0in
\evensidemargin 0in
\textheight 8.5in
\textwidth 6.5in
\topmargin -0.25in

\pagestyle{fancy}
\lhead{\bf BME590.06: Medical Software Design}
\rhead{\bf Palmeri \& Kumar (Fall 2017)}
\cfoot{\thepage}


\begin{document}

\section*{Class Syllabus}

{\bf Instructors:}\\
Dr. Mark Palmeri, M.D., Ph.D.\\
\href{mailto:mark.palmeri@duke.edu}{(mark.palmeri@duke.edu)}\\
Office Hours: TBD (258 Hudson Hall Annex)

Suyash Kumar, CTO kelaHealth, Former Uber Software Engineer\\
\href{mailto:suyash.kumar@duke.edu}{(suyash.kumar@duke.edu)}\\
Office Hours: TBD (TBD)

{\bf Teaching Assistant:}
Arjun Desai\\
\href{mailto:arjun.desai@duke.edu}{(arjun.desai@duke.edu)}\\
Office Hours: TBD (TBD)

{\bf Lecture:} Tues/Thurs 15:05--16:20, 125 Hudson Hall

\subsection*{Course Overview}
Software plays a critical role in almost all medical devices, spanning device control, feedback and algorithmic processing.  This course focuses on software design skills that are ubiquitous in the medical device industry, including software version control, unit testing, fault tolerance, continuous integration testing and documentation.  Experience will be gained in both dynamically- (Python) and statically-typed (C/C++) languages. 

The course will be structured around a project to build an Internet-connected medical device that measures and processes a biosignal, sends it to a web server, and makes those data accessible to a web client / mobile application.  This project will be broken into several smaller projects to develop software design fundamentals.  All project-related work will be done in groups of 3 students.

Prerequisites: Introductory Programming Class (e.g., EGR103)

\section*{Course Objectives}
\begin{multicols}{2}
\begin{itemize}
    \item Define software specifications and constraints
    \item Agile project management
    \begin{itemize}
        \item Team roles
        \item User stories
        \item Backlog, To Do, In Progress, Completed, Blocked ``Issues'' (Kanban)
        \item Weekly sprints
        \item MVPs
    \end{itemize}
    \item Device programming fundamentals
    \begin{itemize}
        \item Review of data types
        \item Analog-to-digital / digital-to-analog conversion
        \item Python (v3.6): numpy, scipy, pandas, scikit
        \item C/C++
        \item Simplified Wrapper and Interface Generator (SWIG)
        \item Data management (variables, references, pointers, ASCII/Unicode/binary data)
        \item Regular expressions (regex)
        \item Compilation, make, cmake
        \item Use of a programming IDE
        \item Debugging
    \end{itemize}
    \item Backend Software Development in the Cloud
    \begin{itemize}
        \item HIPPA Compliance in the cloud
        \item Databases
        \item HTTP \& RESTful APIs
        \item Leverage scalable compute infrastructure in the cloud via Remote Procedure Calls (RPCs)
        \item Call web services from Matlab \& Python
        \item Design \& Implementation of a biomedical web service (Python Flask)
        \item Docker and dependency management intro
        \item SSL and Encryption
        \item Internet of Things (IoT) and cloud connected biomedical device design
        \item Load Balancing and throughput bottlenecks
        \item [TBD] Sockets and streaming data over networks
    \end{itemize}
    \item Software version control (git, GitHub)
    \item Documentation
    \begin{itemize}
        \item Docstrings
        \item Markdown
        \item Sphinx / Doxygen
    \end{itemize}
    \item Testing
    \begin{itemize}
        \item Unit testing
        \item Functional / System testing
        \item Continuous integration (Travis CI)
    \end{itemize}
    \item Fault tolerance (raising exceptions)
    \item Resource profiling (cProfile)
\end{itemize}
\end{multicols}

\subsection*{Fall Semester Class Schedule} 
The course schedule is subject to change depending on progress throughout the semester.
Specific lecture details, along with deliverable due dates,
will be updated on Sakai and this syllabus (which is hosted in a GitHub repository). New due dates will be announced in lecture
and by Sakai announcements that will be emailed to the class.  The following is
a summary of activities this semester:

\begin{longtable}[c]{|l|l|l|}

    \hline 
    
    \textbf{Date} & \textbf{Lecture} & \textbf{Assignment}\\

    \hline

    Tues Aug 29     & Class Introduction, Objectives and Logistics; Git Demo & A01: Git(Hub), Python\\
    Thurs Aug 31    & Git: Repo Setup, Issues, Branching, Pushing/Pulling & In-class exercise\\
    \hline
    Tues Sep 05     & Python virtualenv (conda/pip), Unit Testing & In-class exercise\\
    Thurs Sep 07    & Continuous Integration (Travis CI), Python Types/Exceptions & Complete unit test exercise\\
    \hline
    Tues Sep 12     & Python Fundamentals \& PyCharm & A02: TBD \\
    Thurs Sep 14    & Python: Dictionaries, Numpy Arrays, Binary Data & \\
    \hline
    Tues Sep 19     & Data Types, Application Program Interfaces (API) \& JSON & \\
    Thurs Sep 21    & Modules, Classes, Composition& \\
                    & Classes: Inheritance \& Composition & \\
    \hline
    Tues Sep 26     & Python: docstrings, Sphinx & A03: TBD\\
    Thurs Sep 28    & Python: try/except, logging & \\
    \hline
    Tues Oct 03     & Python: Read/Writing Data (CSV, JSON, HDF5, MATv5) & A04: robust, logging \\
    Thurs Oct 05    & Regular Expressions & \\
    \hline
    Tues Oct 10     & FALL BREAK & \\
    Thurs Oct 12    & Installing SWIG \& C/C++ Compiler & \\
    \hline
    Tues Oct 17     & Python $\rightarrow$ C/C++ \& Static-Compilation & \\
    Thurs Oct 19    & Binary Data Files \& Bit Operations & \\
    \hline
    Tues Oct 24     & Continuous Integration Testing & \\
    Thurs Oct 26    & Regular Expressions & \\
    \hline
    Tues Oct 31     & TBD & A05: TBD \\
    Thurs Nov 02    & Simplified Wrapper \& Interface Generator (SWIG) & \\
    \hline
    Tues Nov 07     & TBD & A06: TBD \\
    Thurs Nov 09    & TBD & \\
    \hline
    Tues Nov 14     & IEC 62304 & Project Unit Tests \\
    Thurs Nov 16    & Project TDD & \\
    \hline
    Tues Nov 28     & Debugging & Working Project Code \\
    Thurs Nov 30    & Presentations: Profiled Code & \\
    \hline
    Tues Dec 05     & Project ``Lab'' & Refactor Project Code \\
    Thurs Dec 07    & Final Project Due & \\
    \hline

\end{longtable}


\subsection*{Attendance}
Lecture attendance and participation is important because you will be working in small groups most of the semester.  Participation in these in-class activities will count for 15\% of
your class grade.  It is very understandable that students will have to miss
class for job interviews, personal reasons, illness, etc.  Absences will
be considered \emph{excused} if they are communicated to Dr. Palmeri and Mr. Kumar at least
48 hours in advance (subject to instructor discretion as an excused absence) or, for illness, through submission of
\href{http://www.pratt.duke.edu/undergrad/policies/3531}{Short Term Illness
    Form (STIF)} {\bf before} class.  Unexcused absences will count
against the participation component of your class grade.  

\subsection*{Textbooks \& References} 
There are no required textbooks for this class.  A variety of online resources will be referenced throughout the semester.  A great resource for an overview of Python:\\
\centerline{\url{https://github.com/jakevdp/WhirlwindTourOfPython}}

\subsection*{Distributed Version Control Software (git)} 
Software management is a ubiquitous tool in any engineering
project, and this task becomes increasingly difficult during group development.
Version control software has many benefits and uses in software development,
including preservation of versions during the development process, the ability
for multiple contributors and reviewers on a project, the ability to tag
``releases'' of code, and the ability to branch code into different functional
branches.  We will be using GitHub (\url{https://github.com}) to
centrally host our git repositories.  Specifically, we will be creating student teams in the Duke BME Design GitHub group (\url{https://github.com/Duke-BME-Design}).  Some guidelines for using your git repositories:

\begin{itemize}
    \item \emph{All} software additions, modifications, bug-fixes, etc.\ need
        to be done in your repository.
    \item The ``Issues'' feature of your repository should be used as a ``to
        do'' list of software-related items, including feature enhancements,
        and bugs that are discovered.
    \item There are several repository management models that we will review in class,
        including branch-development models that needs to be used throughout the semester.
    \item Instructors and teaching assistants will only review code that is committed to your repository (no emailed code!).  
    \item All of the commits associated with your repository are logged with
        your name and a timestamp, and these cannot be modified.  Use
        descriptive commit messages so that your group members, instructors,
        and teaching assistants can figure out what you have done!!  You should not
        need to email group members when you have performed a commit; your
        commit message(s) should speak for themselves.
    \item Code milestones should be properly tagged.
    \item Write software testing routines early in the development process so
        that anyone in your group or an outsider reviewing your code can be
        convinced that it is working as intended.
    \item Modular, modular, modular.
    \item Document!
    \item Make commits small and logical; do them often!
\end{itemize}

We will review working with git repositories in lecture, and feedback
on your software repository will be provided on a regular basis.

\subsection*{Project Details} 
Project details will be discussed in lecture throughout the semester.

\subsection*{Grading} 
The following grading scheme is subject to change as the semester progresses:

\begin{center}
\begin{tabular}{ll}
Participation                           & 15\% \\
Midterm project deliverables            & 55\% \\
Final project                           & 30\% \\
\end{tabular}
\end{center}

\subsection*{Duke Community Standard \& Academic Honor} Engineering is
inherently a collaborative field, and in this class, you are encouraged to work
collaboratively on your projects.  The work that you submit must be the product
of your and your group's effort and understanding.  All resources developed by
another person or company, and used in your project, must be properly
recognized.
 
All students are expected to adhere to all principles of the
\href{http://www.integrity.duke.edu/standard.html}{Duke Community Standard}.
Violations of the Duke Community Standard will be referred immediately to the
Office of Student Conduct.  Please do not hesitate to talk with your instructors
about any situations involving academic honor, especially if it is ambiguous
what should be done.

\end{document}
