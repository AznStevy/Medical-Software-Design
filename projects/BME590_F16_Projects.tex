\documentclass[10pt]{report}
\usepackage{epsf}
\usepackage{amsmath}
\usepackage{amssymb}
\usepackage{palatino}
\usepackage[dvips]{graphics}
\usepackage{fancyhdr}
\usepackage{epsfig}
\usepackage{multirow}
\usepackage{multicol}
\usepackage{cancel}
\usepackage{hyperref}
\usepackage{longtable}
\parindent 0in
\parskip 1ex
\oddsidemargin  0in
\evensidemargin 0in
\textheight 8.5in
\textwidth 6.5in
\topmargin -0.25in

\pagestyle{fancy}
\lhead{\bf BME590.06: Medical Software Design}
\rhead{\bf Palmeri \& Kumar (Fall 2017)}
\cfoot{\thepage}


\begin{document}
\section*{Final Projects}

{\bf DUE:} Friday, 2016-12-16 at 10:00 PM (when our final exam block time would end)

\begin{itemize}

    \item You will be working in your assigned group.
        \footnote{\url{https://gitlab.oit.duke.edu/medical-device-software-design/rw_data}}
        Please create a repository called \verb+project_netid_netid+ (fill
        in netid of each student member).  Be sure to add Dr.~Palmeri
        (\verb+mlp6+) and Brenton (\verb+bnk5+) as a Master access-level member.

    \item Use all good git repository management practices that have been promoted all semester.

    \item Create Issues (that are associated with Milestones) for all
        development tasks on the project, and \textbf{assign a specific group
        member} to each task.  While this is a group project, each group member
        will be graded individually based on their contributions to the
        project, so strive to have even effort distribution, as represented by
        these issues.  \textbf{Be sure to associate commits with specific issues.}

    \item Use all good python coding practices that have been promoted all
        semester, including PEP8 style compliance.

    \item \textbf{At least one ``core'' computational aspect of your codebase
        needs to be implemented in C/C++ and interfaced as an importable module
        into Python using SWIG.}

    \item Create an annotated tag (\verb+v1.0.0+) of your final version.

    \item Choose one of the following project topics:

    \begin{enumerate}
        \item Develop software that identifies the P, R and T events on a per-beat basis in an ECG signal.  This software will:
        \begin{itemize}
            \item Read in time and voltage data from a Matlab (v5) file over a finite period of time,
            \item Display/save a plot of the ECG signal with each P, R, and T event indicated,
            \item Save an output file that stores all of the absolute times for each P, R and T event.
        \end{itemize}
        \item Develop software that augments your ultrasound B-mode image generation assignment to:
        \begin{itemize}
            \item Have an interactive GUI using either Tkinter or Qt,
            \item Choose the JSON and binary datafile to load in the GUI,
            \item Provide the ability for the user to interactively change the logarithmic compression level, TGC, and other layers of image optimization (e.g., histogram equalization) and re-render the image.
            \item Provide a `Save' option to save a PNG, JPG or TIFF file based on auto-discovery of the file extension.
        \end{itemize}
        \item Automated Cervical Cancer Screening Project (please see attached PDF).  This project will involve implementing a Support Vector Machine (SVM), which is greatly facilitated using the \verb+scikit-learn+ package.
        \item In addition to generating B-mode images, ultrasound can be used to generate M-mode images, where data are acquired at the same location serially through time.  These M-mode data are used to estimate motion at that spatial location through time.  A common application of this is to track the motion of ventricular and septal walls of the heart throughout the cardiac cycle.  You will develop software that:
        \begin{itemize}
            \item Reads in metadata in JSON format and RF data from a binary file,
            \item Estimates the localized motion of cardiac ventricular and septal walls in the M-mode data using RF cross correlation and phase-shift estimates in demodulated IQ data,
            \item Plots and saves the localized motion in a developer-decided format.
        \end{itemize}
        \item Group-defined project.  You are welcome to propose your own project.  Please submit a project proposal to me by email that includes:
        \begin{itemize}
            \item Overview of software (clinical application)
            \item Functional specifications
            \item Data to be input
            \item Expected algorithmic implementations
            \item Data to be output
        \end{itemize}
    \end{enumerate}

    \item Test data will be posted for the ECG, B-mode, M-mode and cervical cancer screening projects.

\item Grading criteria:
\emph{You should approach this final project as an opportunity to show a potential future employer an example of your software development skills.}

\begin{itemize}
    \item Git Repository
        \begin{itemize}
            \item Issues/Milestones [10\%]
            \item Commits are discrete, logical changesets [10\%]
            \item Branching \& Merging [5\%]
        \end{itemize}
    \item Modular coding [10\%]
    \item Avoidance of hard-coded variables; robust functional input
        for algorithmic control. [10\%]
    \item Full unit test coverage of all functions, with passing CI
        build\footnote{Dr. Palmeri will enable gitlab runner for your
        repository.} [20\%]
    \item Logging: INFO, DEBUG, ERROR [10\%]
    \item Sphinx documentation for each module/function [10\%]
    \item Handle and raise exceptions [5\%]
    \item Functionality [10\%]
\end{itemize}

\end{itemize}

\end{document}


