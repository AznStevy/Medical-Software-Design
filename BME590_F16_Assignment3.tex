\documentclass[10pt]{report}
\usepackage{epsf}
\usepackage{amsmath}
\usepackage{amssymb}
\usepackage{palatino}
\usepackage[dvips]{graphics}
\usepackage{fancyhdr}
\usepackage{epsfig}
\usepackage{multirow}
\usepackage{multicol}
\usepackage{cancel}
\usepackage{hyperref}
\usepackage{longtable}
\parindent 0in
\parskip 1ex
\oddsidemargin  0in
\evensidemargin 0in
\textheight 8.5in
\textwidth 6.5in
\topmargin -0.25in

\pagestyle{fancy}
\lhead{\bf BME590.06: Medical Software Design}
\rhead{\bf Palmeri \& Kumar (Fall 2017)}
\cfoot{\thepage}


\begin{document}
\section*{Assignment \#3: Heart Rate Monitor (RC2)}

{\bf DUE:} Friday, 2016-10-14 at 17:00.

\begin{itemize}

\item Continue this assignment in your \verb+bme590assignment02+ repository.  None of the items below require unit tests, but you can use git Issues/Milestones as you did in assignment 02.

\item Demonstrate your ability to add an external repository into your repository as a git submodule.  
\begin{itemize}
    \item If you are already using an external package in your code, then make sure that is appropriately added as a submodule.  
    \item If you do note use an external package, then add the following repository as a submodule: \verb+https://gitlab.oit.duke.edu/medical-device-software-design/submodule_test+
\end{itemize}

\item Expand the functionality of your code using \verb+argparse+ to allow a user to specify certain input arguments to your code.  Implement all of these features on a new branch, with each input argument being an independent commit, and only merge into your master branch when complete.
\begin{itemize}
    \item Create an input argument that specifies the binary data filename.
    \item Create input arguments for the thresholds for bradycardia and tachycardia.
    \item Create and input argument (or arguments) that chooses if your heart rate monitor uses (a) just the ECG signal, (b) just the pulse plethysmograph signal, or (c) both the ECG and pulse plethysmograph signals to estimate heart rate.  \emph{Note - this will potentially require a little rework of your code, but hopefully it is modular enough to accommodate this relatively easily.}
    \item Create an input argument that specifies the duration of the multi-minute heart rate averages (not just fixed values of 1 and 5 min).
    \item \emph{For all of these inputs, make sure that you specify input types, help strings, and default values.  There should also be an overall description provided for your code when the} \verb+--help+ \emph{flag is issued.}
\end{itemize}

\item On a new branch, generate Sphinx documentation in your repository using the docstrings for all of your methods.
\begin{itemize}
    \item Make sure the documentation is contained in a \verb+docs/+ subdirectory.
    \item Generate a \verb+Makefile+ to build the documentation.
    \item Locally generate HTML documentation, but do not add the \verb+html/+ subdirectory to your git commit (this will create a lot of files, and it is better practice to simply include the Makefile so an end-user can generate the documentation.)
    \item Merge your documentation branch into your master branch when complete.
\end{itemize}


\item Create a new git annotated tag of `RC2' when you are done with the assignment.

\item Grading criteria:
\begin{itemize}
    \item Git usage (same as assignment 02 guidelines) [25\%]
    \item Git submodule usage [15\%]
    \item \verb+argparse+ implementation [35\%]
    \item Sphinx documentation [25\%]
\end{itemize}




\end{itemize}

\end{document}
